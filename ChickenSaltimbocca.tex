\documentclass[11pt,a4paper]{article}

\usepackage[headheight=100pt]{geometry}
\usepackage{fancyhdr}   % custom headers
\usepackage{lipsum}     % insert dummy 'Lorem ipsum' text
\usepackage{amsmath}    % measurement fractions
\usepackage{multicol}   % multi-column for ingredients
\usepackage{hyperref}   % links
\usepackage{times}      % alternate font

\pagestyle{fancy}
\fancyhf{}              % clear header and footer
\fancyhead[L]{\fontsize{14}{10} \selectfont Chicken Saltimbocca\footnotemark[1]}

\begin{document}

\subsection*{Ingredients}

\begin{multicols}{2}

\begin{itemize}
   \item $ 2 $ chicken fillets
   \item $ 8 $ sage leaves
   \item $ 8 $ slices pancetta or prosciutto
   \item $ \frac {1} {2} $ cup plain flour
\end{itemize}

\columnbreak

\begin{itemize}
   \item $ 1 $ tablespoon olive oil
   \item $ 1 $ tablespoon butter
   \item $ \frac {1} {2} $ cup white wine or sake
   \item $ 1 $ teaspoon Dijon mustard
   \item $ 50 $ grams chilled, cubed butter
\end{itemize}

\end{multicols}

\medskip

\subsection*{Method}

\begin{enumerate}
   \item Cut chicken fillets in half. Cut each half through cross-ways so that you have two thin, flat pieces.
   \item Place a sage leaf on top and cover with a piece of pancetta. 
         Repeat until all pieces are covered (four pieces of chicken for each fillet). 
         Dust lightly with flour.
   \item Heat oil and butter in a large frying pan and place chicken in pan, pancetta side down.
         Cook for a couple of minutes each side, until golden.
         Remove and set aside.
   \item Deglaze the pan by adding the white wine. Gently scrape up any bits from the bottom of the pan with a wooden spoon.
         Add chilled butter, a little at a time, stir until combined and slightly thickened. Stir in Dijon mustard.
         When sauce is ready, return chicken to pan and heat through.
   \item Serve with steamed vegetables.
\end{enumerate}

\footnotetext[1]{Pronounced \textit{saltim'bokka}, Italian for jumps in the mouth.}

\end{document}
