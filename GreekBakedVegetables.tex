\documentclass[11pt,a4paper]{article}

\usepackage[headheight=100pt]{geometry}
\usepackage{fancyhdr}   % custom headers
\usepackage{amsmath}    % measurement fractions
\usepackage{multicol}   % multi column for ingredients
\usepackage{hyperref}   % links
\usepackage{times}      % alternate font
\usepackage{textcomp}   % temperature

\pagestyle{fancy}
\fancyhf{}  % clear header and footer
\fancyhead[L]{\fontsize{14}{10} \selectfont Greek Style Baked Vegetables}

\begin{document}

\subsection*{Ingredients}

\begin{multicols}{2}

\begin{itemize}
  \item potato
  \item onion
  \item cauliflower
  \item capsicum
  \item eggplant
  \item beans
  \item plus other vegetables\dots
\end{itemize}

    \columnbreak{}

\begin{itemize}
  \item olive oil
  \item mixed herbs
  \item garlic
  \item chilli
  \item tomatoes ($\frac{1}{2}$ to blend, $\frac{1}{2}$ to layer on top)
  \item tomato paste
  \item salt and pepper
\end{itemize}

\end{multicols}

\medskip

\subsection*{Method}

\begin{enumerate}
   \item Cube your vegetables.  Pour oil over vegetables, add mixed herbs.  Mix well.
   \item Place into a single layer in large baking tray.  Grill until coloured.
   \item Blend part of the tomatoes, tomato paste, one small chilli and garlic with water.  Pour over the top of the vegetables.  Cover with foil.  Place in a preheated 200\textdegree{} C oven.  After 40 minutes, remove the dish and baste vegetables with their juices.  Uncover, reduce the oven to 150\textdegree{} C and cook for 30 minutes.  Check to make sure the liquid isn't too low.  If you need to, top with a little more water.
   \item Remove from oven when the vegetables are a little charred on top and soft.  Squeeze two fresh lemons over the vegetables.  Let it stand for 10 \textendash 15 minutes before serving.
\end{enumerate}

Modified from \href{https://www.artfrommytable.com/baked-vegetable-medley-greek-style/}{Baked Vegetable Medley (Greek Style)}

\end{document}
